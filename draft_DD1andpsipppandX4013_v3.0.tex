\RequirePackage{lineno}
\documentclass[aps,preprint,superscriptaddress,12pt,tightenlines]{revtex4}
\usepackage{times}
\usepackage{amsmath}
\usepackage{graphicx}
\usepackage{epsfig}
\usepackage{dcolumn}
\usepackage{bm}
\usepackage{color}
\usepackage{overpic}
%%%%%%
\graphicspath{{figures/}}
\newcommand{\br}[1]{\mathcal{B}(#1)}
\newcommand{\zc}{Z_c(3900)}
\newcommand{\zcp}{Z_c(4020)}
\newcommand{\lum}{{\cal L}}
\newcommand{\eff}{\varepsilon}
\newcommand{\BR}{{\cal B}}
\newcommand{\jpc}{J^{PC}}
\newcommand{\pip}{\pi^+}
\newcommand{\pim}{\pi^-}
\newcommand{\piz}{\pi^0}
\newcommand{\etap}{\eta^{\prime}}
\newcommand{\hc}{h_c}
\newcommand{\pphc}{\pi^+\pi^- h_c}
\newcommand{\etacp}{\eta_c(2S)}
\newcommand{\psp}{\psi(2S)}
\newcommand{\pspp}{\psi(3770)}
\newcommand{\jpsi}{J/\psi}
\newcommand{\psift}{\psi(4040)}
\newcommand{\psifto}{\psi(4160)}
\newcommand{\psiftf}{\psi(4415)}
\newcommand{\EE}{e^+e^-}
\newcommand{\MM}{\mu^+\mu^-}
\newcommand{\LL}{\ell^+\ell^-}
\newcommand{\GG}{\gamma\gamma}
\newcommand{\pp}{\pi^+\pi^-}
\newcommand{\ppp}{\pi^+\pi^-\pi^0}
\newcommand{\kk}{K^+K^-}
\newcommand{\ddb}{D\bar{D}}
\newcommand{\ddn}{D^0\bar{D}^0}
\newcommand{\ddc}{D^+D^-}
\newcommand{\ra}{\rightarrow}
\newcommand{\etajpsi}{\eta J/\psi}
\newcommand{\ppjpsi}{\pi^+\pi^- J/\psi}
\newcommand{\jpsipp}{\pi^+\pi^- J/\psi}
\newcommand{\pppsp}{\pi^+\pi^- \psp}
\newcommand{\psppp}{\pi^+\pi^- \psp}
%%%
\parskip=5pt plus 1pt minus 1pt
%%%
%%%%%%%%%%%%%%%%%%%%%%%%%%%%%%%%%%%
\def\Journal#1#2#3#4{{#1} {\bf #2}, #3 (#4)}
\def\IJMP{Int. J. Mod. Phys. A}
\def\NCA{Nuovo Cimento}
\def\NIM{Nucl. Instrum. Methods}
\def\NIMA{Nucl. Instrum. Methods A}
\def\NPB{Nucl. Phys. B}
\def\PLB{Phys. Lett. B}
\def\PRL{Phys. Rev. Lett.}
\def\PRD{Phys. Rev. D}
\def\PRP{Phys. Rep.}
\def\ZPC{Z. Phys. C}
\def\EPJC{Eur. Phys. J. C}
\def\HEPNP{HEP \& NP}
%\tiny \scriptsize \footnotesize \small \normalsize
%\large \Large \LARGE \huge \Huge
%%%%%%%%%%%%%%%%%%%%%%%%%%%%%%%%%%%
\begin{document}
\linenumbers

\graphicspath{{figure/}}
\DeclareGraphicsExtensions{.eps,.png,.ps}
%\hyphenpenalty=0
%\tolerance=1000

\preprint{\vbox{\hbox{}
                \hbox{BESIII draft - V2.0}
                }}

\title{\boldmath
Observation of $e^{+}e^{-}\rightarrow \pi^{+}\pi^{-}\psi(3770)$
and $D\bar{D}_{1}(2420)$}
\author{Y.~Hu}
\author{W.~M.~Song}
\author{C.~Z.~Yuan }
\address{Institute of High Energy Physics, Beijing, 100049, China }
\collaboration{BESIII Collaboration}
\date{\today}

\begin{abstract}

We present a study of $e^{+}e^{-}\to \pp\ddn$, $\pp\ddc$ and the
intermediate processes using data samples at center-of-mass
energies above 4.09~GeV collected with the BESIII detector at
BEPCII. In the $D\bar{D}$ invariant mass spectrum, we observe the
$\psi(3770)$ signal for the first time and the Born cross section
of $e^{+}e^{-}\to \pi^{+}\pi^{-}\psi(3770)$ are measured. We also
search for the possible $Z_{c}$ state in $\pi^{\pm}\psi(3770)$
invariant mass spectrum, no obvious signal is found.  We also
search for the $X(4013)$, the possible heavy-quark-spin-symmetry partner of
the $X(3872)$, in $D\bar{D}$ invariant mass spectrum, no obvious
signal is found. The upper limits of
$\sigma[e^{+}e^{-}\to \rho^{0}X(4013)]\times \mathcal{B}(X\to
D\bar{D})$ are given. In the $D\pi^{+}\pi^{-}$ invariant mass
spectrum, we observe the $D_{1}(2420)$ signal. The Born cross
section of $\EE\to D_{1}(2420)^{0}\bar{D}^{0}+c.c.$ times the
$D_{1}(2420)^{0}\to D^{0}\pi^{+}\pi^{-}$ branching fraction and
that of $\EE\to D_{1}(2420)^{0}\bar{D}^{0}+c.c.$ times the
$D_{1}(2420)^{0}\to D^{*+}\pi^{-}$ branching fraction, $\EE\to D_{1}(2420)^{+}D^{-}+c.c.$ times the
$D_{1}(2420)^{+}\to D^{+}\pi^{+}\pi^{-}$ branching fraction for
center-of-mass energies above 4.3~GeV are measured for the first
time.

\end{abstract}

\pacs{14.40.Rt, 13.20.Gd, 13.66.Bc, 13.40.Hq, 14.40.Pq}

\maketitle

\section{Introduction}

The heavy quarkonia have been studied for more than forty years
for testing and developing the quantum chromodynamics (QCD). On
one hand, some effective theories has been developed to describe
the quarkonium spectroscopy and transition dynamics; on the other
hand, many states with exotic properties ($XYZ$ particles) were
discovered, and some of them are beyond the scope of current
theoretical framework. The rich information on the $XYZ$ particles
may have opened a door though which Quark Confinement can be
understood. To understand these $XYZ$ particles, the knowledge
about the traditional quarkonia is, of course, also of great
importance.

In recent years, several new vector charmonium-like states, the
$Y(4260)$, $Y(4360)$, and $Y(4660)$ have been discovered via their
decays into the hidden-charm $\pi^{+}\pi^{-}J/\psi$ or
$\pi^{+}\pi^{-}\psi(3686)$ final states~\cite{BabarY4260,
BabarY4360, BELLEY4260, BELLEY4360}. The charged $Z_{c}(3900)$ and
$Z_{c}(4050)$ have been observed in $\pi^{\pm}J/\psi$ and
$\pi^{\pm}\psi(3686)$ invariant mass spectrum in the processes
$e^{+}e^{-}\to \pi^{+}\pi^{-}J/\psi$ and
$\pp\psi(3686)$~\cite{zc3900, BELLEY4260, BELLEY4360},
respectively. A natural extension of the analysis is the search
for a similar final state $\EE\to \pi^{+}\pi^{-}\psi(3770)$ and
search for similar charged structure decays into
$\pi^{\pm}\psi(3770)$.

The $\psi(3770)$ is generally assumed to be the $1^{3}D_{1}$
charmonium state with some mixture of the $2^{3}S_{1}$, this makes
the $\psi(3770)$ a bit different from the $J/\psi$ and
$\psi(3686)$ which are $^{3}S_{1}$ dominant. One of the other two
$D$-wave spin-triplet charmonium states, $X(3823)$ or the
$\psi(1^{3}D_{2})$, has been observed in $\EE\to
\pp\psi(1^{3}D_{2})$ at BESIII~\cite{X3823}, so a naive
expectation is that, $\EE\to \pp \pspp$ and $\EE\to \pp
\psi(1^3D_3)$ should have been produced in the BESIII data,
although so far there is no calculation on how large the production
rates are. The $\psi(3770)$ decays dominantly to $D\bar{D}$, which
is also an important decay mode of the
$\psi(1^{3}D_{3})$~\cite{Highercharmonia}, so we study $\EE\to
\pi\pi D\bar{D}$ final states in this analysis.

The $X(3872)$ state was first observed by Belle Collaboration~\cite{X3872}, and
confirmed subsequently by several other
experiments~\cite{CDFX3872,D0X3872,BABARX3872}. Even though it
clearly contains a $c\bar{c}$ pair, the $X(3872)$ does not fit
well within the conventional charmonium spectrum. It could be a
$D\bar{D}^{*}$ molecule with $J^{PC} = 1^{++}$, and within this
picture, the existence of its heavy quark spin symmetry partner,
$X_{2}$ ($J^{PC} = 2^{++}$), an $S$-wave $D^{*}\bar{D}^{*}$ bound
state is predicted~\cite{PRD056004,PRD076006}. Its mass is about
4013~MeV/$c^2$ and it decays dominantly into $D\bar{D}$,
$D\bar{D}^{*}$ and $\bar{D}D^{*}$ in $D$-wave. So $X(4013)$ can
also be produced in $e^{+}e^{-}\to \pi^{+}\pi^{-} D\bar{D}$. The
possible discovery of the $2^{++}$ charmonium-like state would
provide a strong support for the interpretation that the $X(3872)$
is dominantly a $D\bar{D}^{*}$ hadronic
molecule~\cite{DecaywidthX4013}.

Amongst various models to interpret the
$Y(4260)$~\cite{BabarY4260,BELLEY4260}, the authors of
Ref.~\cite{Y4260asDD1} argue that the $Y(4260)$ as a relative
$S$-wave $D_{1}(2420)\bar{D}$ system is able to accommodate nearly all
the present observations of the $Y(4260)$, especially its absence
in various open charm decay channels and the observation of
$Z_{c}(3900)$ in $Y(4260)\to \ppjpsi$. In this model, the
$D_{1}(2420)\bar{D}$ coupling to the $Y(4260)$ is a key piece of
information. Because of $D_{1}(2420)$ decays to $D\pi\pi$ or $D^{*}\pi$,
we can also study it via $\EE\to \pi\pi D\bar{D}$ final state.

In this analysis the $D\bar{D}$ (including $\ddn$ and $\ddc$) pair
are reconstructed and selected with both $D$-mesons reconstructed.
$D^{0}$ is reconstructed in four decay modes ($K^{-}\pi^{+}$,
$K^{-}\pi^{+}\pi^{0}$, $K^{-}\pi^{+}\pi^{+}\pi^{-}$, and
$K^{-}\pi^{+}\pi^{+}\pi^{-}\pi^{0}$) and $D^{+}$ in five decay
modes ($K^{-}\pi^{+}\pi^{+}$, $K^{-}\pi^{+}\pi^{+}\pi^{0}$,
$K^{0}_{S}\pi^{+}$, $K^{0}_{S}\pi^{+}\pi^{0}$, and
$K^{0}_{S}\pi^{+}\pi^{-}\pi^{+}$). $\bar{D}^{0}$ and $D^{-}$ are
reconstructed in the charge conjugate final states of $D^0$ and
$D^+$, respectively. One $\pp$ pair is selected in addition to
those from the $D$ decays.

\section{THE EXPERIMENT AND DATA SETS}

The BEPCII is a double-ring $e^{+}e^{-}$ collider running at
center-of-mass (c.m.) energies ranging from 2.0 to 4.6~GeV. The
BESIII spectrometer hosted at the BEPCII is designed to study
physics in $\tau$-charm energy region~\cite{bepc2,bes3yellow}. It
covers 93\% of 4$\pi$ in geometrical acceptance and it consists of
four main parts. A 43 cylindrical layers cell Helium-gas
based(40\% He, 60\% $C_{3}H_{8}$) drift chamber (MDC) which
provides an average single-hit resolution of 135~$\mu$m and
momentum measurements of charged particles; A plastic scintillator
Time-of-Flight (TOF) system which  provides a time resolution of
80~(110)~ps in the barrel~(end cap) detector; A CsI(Tl)
Electro-Magnetic Calorimeter (EMC) which is used to measure the
energies of photons and electrons and provides an energy
resolution of  2.5\% (5.0\%) in the barrel (end caps); A RPC-based
muon chamber (MUC) with a superconducting magnet providing 1.0~T
magnetic field in the central region of BESIII.

Simulated data are produced by the {\sc geant4}-based Monte Carlo
(MC) simulation software {\sc boost}~\cite{boost}, which includes
the geometric and material description of the BESIII detector, the
detector response and digitization models, as well as the tracking
of the detector running conditions and performance. In order to
optimize the selection criteria, determine the detection
efficiency, signal MC samples with 200,000 events for each process
$e^{+}e^{-}\to \pi^{+}\pi^{-}\psi(3770)$, with $\psi(3770) \to
D\bar{D}$, $e^{+}e^{-}\to \rho^{0}X(4013)$ with
$\rho^{0} \to \pi^{+}\pi^{-}$, $X(4013)\to D\bar{D}$ and $e^{+}e^{-}\to D_{1}(2420)\bar{D}+c.c.$ with $D_{1}(2420)\to
D\pi^{+}\pi^{-}+c.c.$are generated
at each c.m. energy point.

In order to estimate the potential background, inclusive MC
samples including $Y(4260)$ decays, $e^{+}e^{-}\to
D^{(*)}D^{(*)}(\pi)$, initial state radiative (ISR) production of
the vector charmonium states, continuum and QED processes are
generated with {\sc kkmc} at c.m. energy $\sqrt{s}=$ 4.23, 4.26,
4.36, 4.42, and 4.6~GeV with the equivalent luminosity as data.
The known decay modes of the charmonia are generated with {\sc
EvtGen}~\cite{EvtGen} with branching fractions being set to the
world average values~\cite{PDG} and the remaining events associate
with charmonium decays are generated with {\sc
Lundcharm}~\cite{Lundcharm} while other hadronic events are
generated with {\sc pythia}~\cite{PYTHIA}.

In addition, exclusive MC samples with 200,000 events for the
processes $e^{+}e^{-}\to\pi^{+}\pi^{-}D^{0}\bar{D}^{0}$, $e^{+}e^{-}\to\pi^{+}\pi^{-}D^{+}D^{-}$, $e^{+}e^{-}\to D^{*+}D^{*-}$ with $D^{*+}\to
D^{0}\pi^{+}+c.c.$, $e^{+}e^{-}\to D^{*+}\bar{D}^{0}\pi^{-}$ with
$D^{*+}\to D^{0}\pi^{+}+c.c.$, $e^{+}e^{-}\to
D^{*}_{0}(2400)\bar{D}$ with $D^{*}_{0}(2400)\to D\pi+c.c.$,
$e^{+}e^{-}\to D_{1}(2430)^{0}\bar{D}^{0}$ with $D_{1}(2430)^{0}\to
D^{*+}\pi^{-}+c.c.$ and $e^{+}e^{-}\to D^{*}_{2}(2460)^{0}\bar{D}^{0}$ with
$D^{*}_{2}(2460)^{0}\to D^{*+}\pi^{-}+c.c.$are
generated at each c.m. energy to study the possible background
contributions.

\section{EVENT SELECTION AND BACKGROUND ANALYSIS}

Charged tracks are reconstructed from MDC hits within a
polar-angle ($\theta$) acceptance range of $|\cos\theta| <0.93$.
To optimize the momentum measurement, we require that these tracks
be reconstructed to pass within 10~cm of the interaction point in
the beam direction and within 1~cm in the plane perpendicular to
the beam. The TOF and dE/dx information are combined for each
charged track to calculate the particle identification (PID)
probability $Prob_{i}$ ($i=\pi$, $K$) of each particle-type
hypothesis. $Prob_K>Prob_\pi$ is required for a kaon candidate and
$Prob_\pi>Prob_K$ is required for a pion candidate. Tracks used in
reconstructing $K^{0}_{S}$ decays are exempted from these
requirements.

Electromagnetic showers are reconstructed by clustering EMC
crystal energies. Efficiency and energy resolution are improved by
including energy deposits in nearby TOF counters. A photon
candidate is defined as a shower with an energy deposit of at
least 25~MeV in the ``barrel" region ($|\cos(\theta)|<0.8$), or of
at least 50~MeV in the ``end-cap" region
($0.86<|\cos(\theta)|<0.92$). Showers at angles intermediate
between the barrel and the end-cap are not well measured and are
rejected. An additional requirement on the EMC hit timing ($0\leq
T\leq14$ in units 50~ns) suppresses electronic noise and energy
deposits unrelated to the event. To eliminate showers produced by
charged particles, the angle between the shower and nearest
charged track is required to be greater than 20 degrees.

$\pi^{0}$ candidates are reconstructed from pairs of photons with
an invariant mass in the range
$0.115<M_{\gamma\gamma}<0.150$~GeV/$c^{2}$. A one-constraint (1C)
kinematic fit with the mass of the $\pi^{0}$ constraint to the PDG
value~\cite{PDG} is also performed to improve the energy
resolution.

$K^{0}_{S}$ candidates are reconstructed from two oppositely
charged tracks which satisfy $|\cos\theta|<0.93$ for the polar
angle and the distance to the average beam position in beam
direction within 20~cm. For each pair of tracks, a constrained
vertex fit is performed and the resulting track parameters are
used to get invariant mass $M_{\pi\pi}$ which must be in the range
$0.487< M_{\pi\pi} < 0.511$~GeV/$c^{2}$. The $\chi^{2}$ from
vertex fit should be little than 100.

The selected  $K^{\pm}$, $\pi^{\pm}$, $K^{0}_{S}$ and $\pi^{0}$
are used to reconstructed $D$ meson candidates for the
$D^{0}\bar{D}^{0}$ and $D^{+}D^{-}$ double tag. If there is more
than one candidate per possible $D\bar{D}$ pair decay mode, the
one with the average mass $\hat{M} = [M(D) + M(\bar{D}]/2$ closest
to the nominal mass of $D$~\cite{PDG} is chosen. In each event,
one extra negative and one extra positive charged $\pi$ are
required. To reduce background and improve the mass resolution, a
four-constraint (4C) kinematic fit is performed, we constrain the
total 4-momentum of all the charged tracks we selected and good
photon from $\pi^{0}$ to that of the initial $e^{+}e^{-}$ system;
if $\pi^0$ or $K^{0}_S$ exists in the final state, its mass
constraint is also applied. If there are multiple candidates in an
event, the one with the smallest $\chi^{2}$ is chosen. The
$\chi^{2}$ of the kinematic fit is required to be less than 80.

Figure~\ref{D0vsDbar} shows the distributions of $M(D)$ versus
$M(\bar{D})$ we selected with this method at $\sqrt{s}=4.36$~GeV.
The signal region in the $M(D)$ and $M(\bar{D})$ plane is defined
as $-6 < \Delta\hat{M} <10$~MeV/$c^{2}$ and $|\Delta{M}| <
35$~MeV/$c^2$ for $D^{0}\bar{D}^{0}$, $-5 < \Delta\hat{M} <
10$~MeV/$c^{2}$ and $|\Delta{M}| < 25$~MeV/$c^2$ for $D^{+}D^{-}$,
where the $\Delta\hat{M} = \hat{M} - m_D$ and $\Delta{M} =
M(D)-M(\bar{D})$, with $m_D$ being the nominal $D$-meson
mass~\cite{PDG}.

\begin{figure*}[htbp]
  \centering
   \begin{overpic}[width=0.45\textwidth]{plotofPPT/D0vsD0bar.eps}
   \put(80,80){(a)}
   \end{overpic}
   \begin{overpic}[width=0.45\textwidth]{plotofPPT/DpvsDm.eps}
   \put(80,80){(b)}
   \end{overpic}
\caption{Scatter plot of the $D^{0}$ mass versus the $\bar{D}^{0}$
mass (a), and $D^{+}$ mass versus the $D^{-}$ mass (b) at
$\sqrt{s} = 4.36$~GeV. The rectangles show the signal region (red)
and sideband regions (pink).}
  \label{D0vsDbar}
\end{figure*}

In order to suppress the background of $e^{+}e^{-}\to
D^{(*)}\bar{D}^{(*)}(\pi)$, we try to examine if there are $D^{*}$
($\bar{D}^{*}$) signal in $D^0\pi^+$ ($\bar{D}^0\pi^-$)
combination. Here the variable $M(D^{0}\pi^{+})-M(D^{0})+m_{D^0}$
[$M(\bar{D}^{0}\pi^{-})-M(\bar{D}^{0})+m_{D^0}$] is used to get
better mass resolution by eliminating the mass resolution effect
coming from the reconstruction of $D^{0}$ ($\bar{D}^0$). The
distributions of $M(D^{0}\pi^{+})-M(D^{0})+m_{D^0}$ and
$M(\bar{D}^{0}\pi^{-})-M(\bar{D}^{0})+m_{D^0}$ are shown in
Fig.~\ref{fitofDstar}. The criteria
$M(D^{0}\pi^{+})-M(D^{0})+m_{D^0}>2.017$~GeV/$c^{2}$ and
$M(\bar{D}^{0}\pi^{-})-M(\bar{D}^{0})+m_{D^0}> 2.017$~GeV/$c^{2}$
are applied to the processes $e^{+}e^{-}\to
\pi^{+}\pi^{-}\psi(3770)$, $\psi(3770)\to D^{0}\bar{D}^{0}$,
$e^{+}e^{-}\to \pi^{+}\pi^{-}X(4013)$, $X(4013)\to
D^{0}\bar{D}^{0}$ and $e^{+}e^{-}\to
D_{1}(2420)^{0}\bar{D}^{0}, D_{1}(2420)^{0}\to D^{0}\pi^{+}\pi^{-}+c.c. $. The criterion $M(D^{0}\pi^{+})-M(D^{0}) +
m_{D^0}<2.017$~GeV/$c^{2}$ and
$M(\bar{D}^{0}\pi^{-})-M(\bar{D}^{0})+m_{D^0}>2.017$~GeV/$c^{2}$
is applied to the processes $e^{+}e^{-}\to D_{1}(2420)^{0}\bar{D}^{0}, D_{1}(2420)^{0}\to D^{*+}\pi^{-}+c.c.$

\begin{figure}[htbp]
  \centering
   \begin{overpic}[width=0.45\textwidth]{plotofPPT/fitofDpipm_4360.eps}
   \put(80,60){(a)}
   \end{overpic}
   \begin{overpic}[width=0.45\textwidth]{plotofPPT/fitofDbarpimm_4360.eps}
   \put(80,60){(b)}
   \end{overpic}
\caption{The distribution of $M(D^{0}\pi^{+})$ (a) and
$M(\bar{D}^{0}\pi^{-})$ (b) at $\sqrt{s}= 4.36$~GeV.}
  \label{fitofDstar}
\end{figure}

 We use the inclusive MC sample to investigate possible backgrounds. There is
no similar peaks found near 3.773~GeV/$c^{2}$ and
4.013~GeV/$c^{2}$ in the $D\bar{D}$ invariant mass distribution,
nor similar peak near 2.42~GeV/$c^{2}$ in the $D\pi\pi$ invariant
mass distribution. From a study of the MC samples of highly
excited $D$ states, we find that only process $e^{+}e^{-}\to
D_{2}^{*}(2460)^{0}\bar{D}^{0}$, $D_{2}^{*}(2460)^{0}\to D^{*}\pi$
can produce a peak near 2.46~GeV/$c^{2}$ in the signal region of $D_{1}(2420)^{0}$ in the invariant mass distribution of
$D^{0}\pi\pi$. So we consider the process $e^{+}e^{-}\to
D_{2}^{*}(2460)^{0}\bar{D}^{0}$, $D_{2}^{*}(2460)^{0}\to D^{*+}\pi^{-}$
as a component of the background of the process $e^{+}e^{-}\to
D_{1}(2420)^{0}\bar{D}^{0}$, $D_{1}(2420)^{0}\to D^{*+}\pi^{-}$.


\section{SIGNAL EXTRACTION}


\subsection{$e^{+}e^{-}\to \pi^{+}\pi^{-}\psi(3770)$}
After imposing all of the above requirements, the $\ddb$ invariant mass distributions are shown in
Fig.~\ref{FitDDbar}. A peak at around 3.773~GeV/$c^{2}$ can be seen
in the $D\bar{D}$ invariant mass distribution, but there is no evidence
for the possible $\psi(1^{3}D_{3})$.

\begin{figure}[t]
  \centering
   \begin{overpic}[width=0.31\textwidth]{plotofPPT/DDbarfit_D0andDpbin_4360.eps}
   \put(80,60){(a)}
   \end{overpic}
   \begin{overpic}[width=0.31\textwidth]{plotofPPT/DDbarfit_D0andDpbin_4420.eps}
   \put(80,60){(b)}
   \end{overpic}
   \begin{overpic}[width=0.31\textwidth]{plotofPPT/DDbarfit_D0andDpbin_4600.eps}
   \put(80,60){(c)}
   \end{overpic}
   \caption{Fits to $D\bar{D}$ invariant
mass distributions at $\sqrt{s} = 4.36$ (a), 4.42
(b), and 4.60~GeV (c). Dots with error bars
are data, the (blue) solid curves are the fit results, the (red) solid curves are $\pspp$ signal, the (pink)
dashed lines are $D_{1}(2420)\bar{D} + c.c.$ background, the blue dash line is are $\pi\pi D\bar{D}$ background and the (green)
histograms are sideband backgrounds. }
  \label{FitDDbar}
\end{figure}


\begin{figure}[t]
  \centering
\begin{overpic}[width=0.31\textwidth]{plotofPPT/pipDDbar_4360.eps}
   \put(80,60){(a1)}
   \end{overpic}
   \begin{overpic}[width=0.31\textwidth]{plotofPPT/pipDDbar_4420.eps}
   \put(80,60){(b1)}
   \end{overpic}
   \begin{overpic}[width=0.31\textwidth]{plotofPPT/pipDDbar_4600.eps}
   \put(80,60){(c1)}
   \end{overpic}
   \begin{overpic}[width=0.31\textwidth]{plotofPPT/pimDDbar_4360.eps}
   \put(80,60){(a2)}
   \end{overpic}
   \begin{overpic}[width=0.31\textwidth]{plotofPPT/pimDDbar_4420.eps}
   \put(80,60){(b2)}
   \end{overpic}
   \begin{overpic}[width=0.31\textwidth]{plotofPPT/pimDDbar_4600.eps}
   \put(80,60){(c2)}
   \end{overpic}
   \caption{The M($\pi^{+}\psi(3770)$) and M($\pi^{-}\psi(3770)$) distributions at $\sqrt{s} = 4.36$ (a1,a2), 4.42 (b1,b2), and 4.60~GeV (c1,c2). The dots with error bars are data, the (green) histograms are $\pspp$ signal MC, the (red) histograms are inclusive backgrounds.}
  \label{Mpipmpsipp}
\end{figure}

To determine the $e^{+}e^{-}\to \pi^{+}\pi^{-}\psi(3770)$ signal yields,
an unbinned maximum likelihood fit to the $M(D\bar{D})$ is performed, the
signal is described with the MC simulated shape, and the
background component includes $\bar{D}D_{1}(2420)+c.c.$ (described
with MC simulated shape), $\pi\pi D\bar{D}$ (described
with MC simulated shape) and none-$D\bar{D}$ background (described
with the $D\bar{D}$ sideband events). In the fit, the signal yield
and the number of $\bar{D}D_{1}(2420)+c.c.$, $\pi\pi D\bar{D}$ backgrounds are free
parameters, and the the number of none-$D\bar{D}$ background is
fixed to the sideband expectation. The signal yields at $\sqrt{s}$
= 4.36, 4.42, 4.6~GeV are $71.3\pm 20.5$, $140.4\pm 23.9$, $4.7\pm
8.4$, respectively, and the statistical significance of the signal
are determined to be $3.8\sigma$, $6.8\sigma$ and $0\sigma$,
respectively, by comparing the log-likelihood values with and
without signal assumption and taking the change of the number of
degree of freedom into account. The fit results are listed in Table~\ref{RCFDDbar}.

With the same method, we also study the data samples taken at
other c.m. energies, but no obvious signals are observed at each
individual energy point. For these data samples, a 90\% confidence
level (C.L.) upper limit on the number of signal events is set
with a Bayesian method~\cite{upperlimit}. In this method, we fit
the $M(D\bar{D})$ distribution with the number of signal events
fixed from 0 to $n$ to get a series of likelihood values. The
upper limit is determined by examining the number of signal events
corresponding to 90\% of the likelihood distribution as a function
of number of signal events. The results are listed in
Table~\ref{RCFDDbar}.

We also check the possible structure in the $\pi^{\pm}\psi(3770)$ invariant
mass distribution as shown in the Fig.~\ref{Mpipmpsipp}, but there is no significant structure been observed.


\subsection{$e^{+}e^{-}\to \pi^{+}\pi^{-}X_{2}(4013)$}

We check the $X(4013)$ signal in high $\ddb$ mass region. After imposing
all of the above requirements, the $\pi^{+}\pi^{-}$ and high mass region of $\ddb$ invariant
mass distributions are shown in Fig.~\ref{FitDDbar_X}. there is no
obvious signal at around 4.013~GeV/$c^{2}$. We try
to fit with MC simulated signal shape, with an 3rd polynomial
background. The signal yields are $1.81\pm 1.49$, $0.00\pm 0.88$,
$3.11\pm 5.71$ with the statistical significance of $1.6\sigma$,
$0\sigma$, and $1.2\sigma$, at $\sqrt{s}= 4.36$, 4.42, and
4.60~GeV, respectively. The fits are shown in Fig.~\ref{FitDDbar_X}.
Because the signals are not significant, 90\% C.L. upper limits on
the numbers of signal events are set with the Bayesian
method~\cite{upperlimit}. The results are listed in
Table~\ref{RCFX4013}.




\begin{figure}[htbp]
   \centering
   \begin{overpic}[width=0.31\textwidth]{plotofPPT/pipim_X_4360.eps}
   \put(80,60){(a1)}
   \end{overpic}
   \begin{overpic}[width=0.31\textwidth]{plotofPPT/pipim_X_4420.eps}
   \put(80,60){(b1)}
   \end{overpic}
   \begin{overpic}[width=0.31\textwidth]{plotofPPT/pipim_X_4600.eps}
   \put(80,60){(c1)}
   \end{overpic}
   \begin{overpic}[width=0.31\textwidth]{plotofPPT/fit_of_DDbarm_4360_8MeV_poly.eps}
   \put(80,60){(a2)}
   \end{overpic}
   \begin{overpic}[width=0.31\textwidth]{plotofPPT/fit_of_DDbarm_4420_8MeV_poly.eps}
   \put(80,60){(b2)}
   \end{overpic}
   \begin{overpic}[width=0.31\textwidth]{plotofPPT/fit_of_DDbarm_4600_8MeV_poly.eps}
   \put(80,60){(c2)}
   \end{overpic}
   \caption{The M($\pi^{+}\pi^{-}$) distributions(first row) and fits to high mass region of $D\bar{D}$ (second row) invariant
mass distributions at $\sqrt{s} = 4.36$ (left column), 4.42 (middle column), and 4.60~GeV (right column). For the first row, the dots with error bars are data, the (pink) histograms are $X_{2}(4013)$ signal MC, the (green) histograms are $\pspp$ signal MC, the (cyan) histograms are $\bar{D}D_{1}(2420)$ signal MC, the (red) histograms are inclusive backgrounds, the shaded histograms are sideband backgrounds.  For the second row, dots with error bars
are data, the (blue) solid curves are the fit results. the (green) dash line are the exponential backgrounds, and the (pink) solid curve are the signal.}
  \label{FitDDbar_X}
\end{figure}


\begin{figure}[htbp]
  \centering
   \begin{overpic}[width=0.31\textwidth]{plotofPPT/Dpipifit_D0unbin_4360.eps}
   \put(80,60){(a1)}
   \end{overpic}
   \begin{overpic}[width=0.31\textwidth]{plotofPPT/Dpipifit_D0unbin_4420.eps}
   \put(80,60){(b1)}
   \end{overpic}
   \begin{overpic}[width=0.31\textwidth]{plotofPPT/Dpipifit_D0unbin_4600.eps}
   \put(80,60){(c1)}
   \end{overpic}
   \begin{overpic}[width=0.31\textwidth]{plotofPPT/Dpipifit_D0starunbin_4360.eps}
   \put(80,60){(a2)}
   \end{overpic}
   \begin{overpic}[width=0.31\textwidth]{plotofPPT/Dpipifit_D0starunbin_4420.eps}
   \put(80,60){(b2)}
   \end{overpic}
   \begin{overpic}[width=0.31\textwidth]{plotofPPT/Dpipifit_D0starunbin_4600.eps}
   \put(80,60){(c2)}
   \end{overpic}
   \begin{overpic}[width=0.31\textwidth]{plotofPPT/Dpipifit_Dpunbin_4360.eps}
   \put(80,60){(a3)}
   \end{overpic}
   \begin{overpic}[width=0.31\textwidth]{plotofPPT/Dpipifit_Dpunbin_4420.eps}
   \put(80,60){(b3)}
   \end{overpic}
   \begin{overpic}[width=0.31\textwidth]{plotofPPT/Dpipifit_Dpunbin_4600.eps}
   \put(80,60){(c3)}
   \end{overpic}
\caption{Fits to $D^{0}\pi^{+}\pi^{-}$
(first row), $D^{*+}\pi^{-}$(second row) and $D^{+}\pi^{+}\pi^{-}$ (third row) invariant
mass distributions at $\sqrt{s} = 4.36$ (left column), 4.42
(middle column), and 4.60~GeV (right column). Dots with error bars
are data, the (blue) solid curves are the fit results, the (pink)
solid curves are  $D_{1}(2420)$ signal, the (pink) dashed lines are
$\pp\psi(3770)$ backgrounds,  the (green) histograms are sideband backgrounds,
the (cyan) solid curves are $D^{*}\bar{D}\pi+c.c.$ backgrounds and
the (cyan) dash line are $D_{2}^{*}(2460)^{0}\bar{D}+c.c.$ backgrounds,}
  \label{FitD1}
\end{figure}

\begin{table}[htbp]
\caption{Results on $e^{+}e^{-}\to \pi^{+}\pi^{-}\psi(3770)$.
Shown in the table are the number of observed signal events
$N^{\rm obs}$, the integrated luminosity $L_{\rm int}$, the
radiative correction factor $1+\delta^{r}$, vacuum polarization
correction factor $1+\delta^{v}$, the summation for product of the
branching fraction and efficiency $\sum_{i,j}\epsilon_{i,j}
\mathcal{B}_{i}\mathcal{B}_{j}$ (the left are for $\ddn$ and the
right are for $\ddc$ modes), the Born cross section $\sigma^{B}$.
The upper limit are at the 90\% C.L.} \label{RCFDDbar}
\begin{tabular}{c c c c c c c c}
\hline \hline
    $\sqrt{s}$ (MeV)      &$N^{\rm obs}$
    &$L_{\rm int}({\rm pb}^{-1})$  &$1+\delta^{r}$
    &$1+\delta^{v}$  &$\sum_{i,j}\epsilon_{i,j} \mathcal{B}_{i}\mathcal{B}_{j}$
    &$\sigma^{B}$ (pb) \\
    \hline
    4085.4&      0.0$\pm$1.0      &52.6    &0.80 &1.052  &0.0078 $\mid$ 0.0093   &0.0$\pm$54.7$\pm$5.4  &0$\sigma$ \\
     4188.6&      0.0$\pm$0.7     &43.1    &0.85 &1.056  &0.0494 $\mid$ 0.0465   &0.0$\pm$5.5$\pm$1.1    &0$\sigma$ \\
     4207.7&      0.0$\pm$1.2     &54.6    &0.85 &1.057  &0.0528 $\mid$ 0.0500   &0.0$\pm$6.7$\pm$3.2  &0$\sigma$ \\
     4217.1&      0.0$\pm$0.7     &54.1    &0.85 &1.057  &0.0544 $\mid$ 0.0518   &0.0$\pm$3.7$\pm$4.6  &0$\sigma$ \\
     4226.3&      14.3$\pm$8.6    &1047.3  &0.84 &1.056  &0.0554 $\mid$ 0.0526   &4.2$\pm$2.5$\pm$1.4    &1.8$\sigma$ \\
     4241.7&      0.0$\pm$0.95    &55.6    &0.84 &1.056  &0.0567 $\mid$ 0.0534   &0$\pm$5.1$\pm$3.3  &0$\sigma$ \\
     4258.0&      30.7$\pm$9.9    &825.7   &0.85 &1.054  &0.0563 $\mid$ 0.0544   &11.7$\pm$3.8$\pm$2.1   &3.54$\sigma$\\
     4307.9&      0.0$\pm$2.5     &44.9    &0.91 &1.052  &0.0511 $\mid$ 0.0499   &0.0$\pm$16.0$\pm$1.5   &0$\sigma$ \\
     4358.3&      71.3$\pm$20.5   &539.8   &0.94 &1.051  &0.0425 $\mid$ 0.0413   &43.1$\pm$12.4$\pm$6.6  &3.8$\sigma$ \\
     4387.4&      15.1$\pm$6.4    &55.2    &0.95 &1.051  &0.0387 $\mid$ 0.0377   &100.7$\pm$42.7$\pm$13.7 &2.9$\sigma$\\
     4415.6&      140.4$\pm$23.9  &1028.9  &0.95 &1.053  &0.0348 $\mid$ 0.0337   &55.7$\pm$9.5$\pm$10.4  &6.8$\sigma$\\
     4467.1&      0.8$\pm$3.8     &109.9   &0.96 &1.055  &0.0290 $\mid$ 0.0288   &3.8$\pm$18.1$\pm$6.2  &0.1$\sigma$\\
     4527.1&      0.0$\pm$0.7     &110.0   &0.96 &1.055  &0.0250 $\mid$ 0.0241   &0.0$\pm$3.6$\pm$1.1    &0$\sigma$\\
     4574.5&      3.6$\pm$3.0     &47.7    &0.96$\pm$0.02 &1.055 $\mid$ 0.0229   &0.0227$\pm$0.0003   &46.6$\pm$38.8$\pm$29.8 &1.5$\sigma$\\
     4599.5&      4.7$\pm$8.4     &566.9   &0.96$\pm$0.02 &1.055 $\mid$ 0.0221   &0.0226$\pm$0.0003   &6.3$\pm$10.9$\pm$6.8   &0.6$\sigma$\\
    \hline \hline
\end{tabular}
\end{table}



\begin{table}[htbp]
\caption{Results on $e^{+}e^{-}\to \rho^{0}X(4013)$. The symbols
are the same as defined in Table~\ref{RCFDDbar}, the last column
is the Born cross section of $e^{+}e^{-}\to \rho^{0}X(4013)$,
$\sigma^{B}$, times branching fraction of $X(4013)\to D\bar{D}$.
The upper limits are at the 90\% C.L.} \label{RCFX4013}
\begin{tabular}{c c c c c c c c c c c}
\hline \hline    $\sqrt{s}$ (MeV)  &$N^{\rm obs}$  &$L_{\rm int}({\rm pb}^{-1})$
&$1+\delta^{r}$  &$1+\delta^{v}$  &$\sum_{i,j}\epsilon\mathcal{B}_{i}\mathcal{B}_{j}$
&$\sigma^{B}\cdot\mathcal{B}_{X(4013)\to D\bar{D}}$ (pb)  \\
    \hline
     4358.3     &1.6$\pm$1.5($<$5.3) &539.8          &0.912  &1.051  &0.021$\mid$0.016  &$<$7.3\\
     4415.6     &0.0$\pm$0.9($<$3.9) &1030.0         &0.923  &1.053  &0.050$\mid$0.031  &$<$1.3\\
     4599.5     &6.6$\pm$5.9($<$16.4)&567.7          &0.948  &1.055  &0.060$\mid$0.038  &$<$7.3\\
    \hline \hline
\end{tabular}
\end{table}

\subsection{$e^{+}e^{-} \rightarrow D_{1}(2420)\bar{D}$}

After imposing all of the above requirements, the $D\pi\pi$ invariant mass distributions are shown in
Fig.~\ref{FitD1}. A peak at around 2.42~GeV/$c^{2}$ can
be seen in the $D\pi\pi$ invariant mass distribution.

To extract the signal yield of $e^{+}e^{-}\to
\bar{D}^{0}D_{1}(2420)^{0}+c.c.$, the $M(D^0\pp)$ distribution is
fitted with a signal shape from the MC simulation convoluted with
a Gaussian function to represent shift of the mass and the mass
resolution difference between data and MC simulation, and a
background component includes $\pi^{+}\pi^{-}\psi(3770)$,
$\pi^{+}\pi^{-} D^{0}\bar{D}^{0}$ and
non-$D\bar{D}$ background. In the fit, the number of
non-$D\bar{D}$ background is fixed to the sideband expectation,
and those of the signal events and the other background events are
free. The signal yields at $\sqrt{s} = 4.36$, 4.42, and 4.6~GeV
are $114.1\pm 32.7$, $215.5\pm 37.1$, and $48.4\pm 16.1$ with the
statistical significance of $3.4\sigma$, $6.2\sigma$, and
$3.5\sigma$, respectively. The fit results are list in the Table~\ref{RCFD0D1}.  With the same
method, we also study the data samples taken at other c.m.
energies, no significant signals are observed at any individual
energy point and upper limit on the number of signal events are
set with the Bayesian method~\cite{upperlimit}. The results are
listed in Table~\ref{RCFD0D1}.


\begin{table}[!htbp]
\caption{The results on $e^{+}e^{-}\to
D_{1}(2420)^{0}\bar{D}^{0}+c.c.$ times branching fraction of $D_{1}(2420)^{0}\rightarrow D^{0}\pi^{+}\pi^{-}+c.c.$. The symbols are the same as
defined in Table~\ref{RCFDDbar}, the last column is the Born cross
section of $e^{+}e^{-}\to D_{1}(2420)^{0}\bar{D}^{0}+c.c.$,
$\sigma^{B}$, times branching fraction of $D_{1}(2420)^{0}\to
D^{0}\pi^{+}\pi^{-}$. The upper limits are at the 90\% C.L. }
\label{RCFD0D1}
\begin{tabular}{c c c c c c c c}
\hline \hline
    $\sqrt{s}$ (GeV)      &$N^{\rm obs}$  &$L_{\rm int}({\rm pb}^{-1})$  &$1+\delta^{r}$  &$1+\delta^{v}$  &$\sum_{i,j}\epsilon_{i,j} \mathcal{B}_{i}\mathcal{B}_{j}$     &$\sigma^{B}\times\mathcal{B}$(pb) &S \\
    \hline
     4307.9&      2.31$\pm$1.7    &44.90  &0.87$\pm$0.03 &1.052  &0.0527$\pm$0.0005   &12.4$\pm$9.1$\pm$1.4   &1.5$\sigma$ \\
     4358.3&      85.5$\pm$27.5   &539.8  &0.94$\pm$0.02 &1.051  &0.0440$\pm$0.0005   &42.4$\pm$16.2$\pm$4.2  &3.4$\sigma$ \\
     4387.4&      11.1$\pm$7.9    &55.2   &0.95$\pm$0.02 &1.051  &0.0399$\pm$0.0004   &58.9$\pm$45.6$\pm$5.8  &1.5$\sigma$ \\
     4415.6&      193.7$\pm$37.5  &1030.0 &0.94$\pm$0.02 &1.053  &0.0310$\pm$0.0004   &72.2$\pm$13.8$\pm$6.5  &6.2$\sigma$ \\
     4467.1&      8.25$\pm$5.7     &110.3  &0.92$\pm$0.02 &1.055  &0.0296$\pm$0.0004  &30.3$\pm$19.8$\pm$2.7  &1.7$\sigma$ \\
     4527.1&      5.4$\pm$2.9     &110.5  &0.94$\pm$0.01 &1.055  &0.0264$\pm$0.0004   &21.7$\pm$11.3$\pm$2.8  &1.2$\sigma$ \\
     4574.5&      2.3$\pm$3.6     &47.7   &0.94$\pm$0.01 &1.055  &0.0240$\pm$0.0003   &23.6$\pm$34.1$\pm$3.1  &0.8$\sigma$ \\
     4599.5&      43.5$\pm$15.1   &567.7  &0.94$\pm$0.01 &1.055  &0.0232$\pm$0.0003   &38.6$\ pm$14.3$\pm$5.1  &3.5$\sigma$ \\
    \hline \hline
\end{tabular}
\end{table}




We do similar fit to M($D^{*+}\pi^{-}$)  to $M(D^0\pp)$ with a
 signal shape from the MC simulation convoluted with
a Gaussian function to represent shift of the mass and the mass
resolution difference between data and MC simulation, and a
background component includes
$D^{*}\bar{D}\pi+c.c.$, $D_{2}^{*}(2460)^{0}\bar{D}+c.c.$ and
non-$D\bar{D}$ background. The signal yields at $\sqrt{s} = 4.36$, 4.42, and 4.6~GeV
are $19.3\pm 12.8$, $37.1\pm 16.5$, and $11.9\pm 8.7$ with the
statistical significance of $1.6\sigma$, $2.4\sigma$, and
$1.5\sigma$, respectively. The fit results are list in the Table~\ref{RCFD0starD1}.  With the
same method, set the upper limit on the number of signal events as
the signal is insignificant. The results are also listed in
Table~\ref{RCFD0starD1}.


\begin{table}[htbp]
\caption{Same as Table~\ref{RCFD0starD1} for the process $e^{+}e^{-}\to
D_{1}(2420)^{0}\bar{D}^{0}+c.c.$ times branching fraction of $D_{1}(2420)^{0}\rightarrow D^{*+}\pi^{-}+c.c.$.}
\label{RCFD0starD1}
\begin{tabular}{c c c c c c c c}
\hline \hline
    $\sqrt{s}$(GeV)      &$N^{obs}$ &$L_{\rm int}({\rm pb}^{-1})$  &$1+\delta^{r}$  &$1+\delta^{v}$  &$\sum_{i,j}\epsilon_{i,j} \mathcal{B}_{i}\mathcal{B}_{j}$     &$\sigma^{B}\times\mathcal{B}$(pb) &S\\
    \hline
     4307.9&      0.0$\pm$1.2   &44.90   &0.9138  &1.052  &0.0641$\pm$0.0005   &0$\pm$6.6$\pm$3.3      & 0$\sigma$ \\
     4358.3&      78.4$\pm$24.8 &539.8   &0.9418  &1.051  &0.0551$\pm$0.0005   &39.1$\pm$12.4$\pm$3.3  & 3.2$\sigma$ \\
     4387.4&      0.0$\pm$3.8   &55.2    &0.9473  &1.051  &0.0504$\pm$0.0005   &0$\pm$20.5$\pm$3.3     & 0$\sigma$ \\
     4415.6&      98.4$\pm$28.0 &1030.0  &0.9516  &1.053  &0.0399$\pm$0.0004   &35.9$\pm$10.2$\pm$2.1  & 3.7$\sigma$ \\
     4467.1&      2.0$\pm$7.3   &110.3   &0.9554  &1.055  &0.0391$\pm$0.0004   &7.1$\pm$25.7$\pm$0.4   & 1.1$\sigma$\\
     4527.1&      2.3$\pm$1.9   &110.5   &0.9597  &1.055  &0.0350$\pm$0.0004   &8.8$\pm$7.3$\pm$0.8    & 0.8$\sigma$ \\
     4574.5&      4.2$\pm$2.9   &47.7    &0.9631  &1.055  &0.0318$\pm$0.0004   &41.4$\pm$28.6$\pm$3.6  & 1.1$\sigma$ \\
     4599.5&      21.3$\pm$11.6 &567.7   &0.9651  &1.055  &0.0312$\pm$0.0004   &18.0$\pm$9.8$\pm$1.6   & 2.1$\sigma$ \\
    \hline \hline
\end{tabular}
\end{table}


We do similar fit to $M(D^+\pp)$ as to $M(D^0\pp)$ with a signal
component and a background component includes
$\pi^{+}\pi^{-}\psi(3770)$, $\pi^{+}\pi^{-} D^{+}D^{-}$  and non-$D\bar{D}$ background. The
signal yields at $\sqrt{s} = 4.36$, 4.42, and 4.6~GeV are $78.4\pm
24.8$, $98.4\pm 28.0$, and $21.3\pm 1.2$ with the statistical
significances of $3.2\sigma$, $3.7\sigma$, and $2.1\sigma$,
respectively.  The fit results are shown in the
Fig.~\ref{FitDDbar} and listed in Table~\ref{RCFDpD1}.  With the
same method, set the upper limit on the number of signal events as
the signal is insignificant. The results are also listed in
Table~\ref{RCFDpD1}.

\begin{table}[htbp]
\caption{Same as Table~\ref{RCFDpD1} for the process
$e^{+}e^{-}\to D_{1}(2420)^{+}D^{-}+c.c.$ times branching fraction of $D_{1}(2420)^{+}\rightarrow D^{+}\pi^{+}\pi^{-}+c.c.$.}
 \label{RCFDpD1}
\begin{tabular}{c c c c c c c c}
\hline \hline
 $\sqrt{s}$ (MeV)      &$N^{\rm obs}$ &$L_{\rm int}({\rm pb}^{-1})$
 &$1+\delta^{r}$  &$1+\delta^{v}$  &$\sum_{i,j}\epsilon_{i,j} \mathcal{B}_{i}\mathcal{B}_{j}$
 &$\sigma^{B}\cdot\mathcal{B}_{D_{1}(2420)^{+}\to D^{+}\pi^{+}\pi^{-}}$ (pb) \\
    \hline
     4307.9&      0.8$\pm$1.4($<$5.2)  &44.90   &0.9138  &1.052  &0.0631   &4.4$\pm$7.3$\pm$0.4($<$27.7)    \\
     4358.3&      94$\pm$19($<$110)    &539.8   &0.9418  &1.051  &0.0540   &46.7$\pm$9.4$\pm$4.5($<$54.6)           \\
     4387.4&      6.8$\pm$4.1($<$13.6) &55.2    &0.9473  &1.051  &0.0488   &36.0$\pm$21.6$\pm$3.5($<72.3$) \\
     4415.6&      129$\pm$26($<$151.2) &1030.0  &0.9516  &1.053  &0.0397   &47.3$\pm$9.4$\pm$8.1($<$55.4)           \\
     4467.1&      7.0$\pm$6.1($<15.5$) &110.3   &0.9554  &1.055  &0.0382   &24.4$\pm$21.2$\pm$4.2($<$53.8)   \\
     4527.1&      2.7$\pm$2.2($<6.8$)  &110.5   &0.9597  &1.055  &0.0345   &10.3$\pm$8.2$\pm$3.2($<$25.8)   \\
     4574.5&      4.6$\pm$2.3($<$9.1)  &47.7    &0.9631  &1.055  &0.0319   &42.6$\pm$21.2$\pm$13.3($<$85.1)   \\
     4599.5&      22.8$\pm$7.5($<$33.3)&567.7   &0.9651  &1.055  &0.0306   &18.3$\pm$6.0$\pm$5.7($<$26.7)     \\
    \hline \hline
\end{tabular}
\end{table}

\section{CROSS SECTION RESULTS}

The Born cross section of $e^{+}e^{-}\to \pi^{+}\pi^{-}\psi(3770)$
is calculated with
\begin{equation}
  \sigma^{B} = \frac{N^{\rm obs}}{\mathcal{L}_{\rm int}
 (1+\delta^{r}) (1+\delta^{v}) (\mathcal{B}_{\psi(3770)\to D^{0}\bar{D}^{0}}
 \sum_{i,j}\epsilon_{i,j} \mathcal{B}_{i}\mathcal{B}_{j}+
 \mathcal{B}_{\psi(3770)\to D^{+}D^{-}} \sum_{k,l}\epsilon_{k,l}\mathcal{B}_{k}\mathcal{B}_{l})},
 \label{xspspp}
\end{equation}
where $N^{\rm obs}$ is the number of observed events;
$\mathcal{L}_{\rm int}$ is the integrated luminosity;
$\epsilon_{i,j}$ is the selection efficiency for $e^{+}e^{-} \to
\pi^{+}\pi^{-}\psi(3770)$, $\psi(3770)\to D^{0}\bar{D}^{0}$,
$D^{0}\to i$, $\bar{D}^{0}\to j)$; $\mathcal{B}_{\pspp\to \ddn}$
and $\mathcal{B}_{\pspp\to \ddc}$ are the branching
fractions~\cite{PDG}; $\mathcal{B}_{i}$ ($\mathcal{B}_{j}$) is the
branching fraction for $D^{0}\to i$ ($\bar{D}^{0}\to j$) (the same
applies to $\epsilon_{k,l}$ and $\mathcal{B}_{k}$
($\mathcal{B}_{l}$) for charged mode) from PDG~\cite{PDG};
$(1+\delta^{v})$ is the vacuum polarization factor~\cite{vpcalculation},
and $(1+\delta^{r})$ is the radiative correction factor which is defined as below:

\begin{equation}
    (1+\delta^{r})= \frac{\sigma^{obs}}{\sigma^{B}} = \frac{\int \,\sigma^{B}(s(1-x))F(x,s)d x}{\sigma^{B}(s)}.
\end{equation}

 \noindent Here, $F(x,s)$ is the radiator function, which is from QED calculation~\cite{QEDcalculation} with an accuracy of 0.1\%; and $\sigma^{B}(s)$ is energy dependent Born cross section of $e^{+}e^{-}\to \pi^{+}\pi^{-}\psi(3770)$.  The Born cross
sections are listed in Table~\ref{RCFDDbar} and shown in Fig.~\ref{Cpipipsi3770}.

\begin{figure}[htbp]
  \centering
  \begin{overpic}[width=0.45\textwidth]{plotofPPT/crosssection_psi3770total.eps}
   \put(80,60){(a)}
   \end{overpic}
   \begin{overpic}[width=0.45\textwidth]{plotofPPT/crosssection_D0D1_total.eps}
   \put(80,60){(b)}
   \end{overpic}
   \begin{overpic}[width=0.45\textwidth]{plotofPPT/crosssection_D0starD1_total.eps}
   \put(80,60){(c)}
   \end{overpic}
   \begin{overpic}[width=0.45\textwidth]{plotofPPT/crosssection_DpD1_total.eps}
   \put(80,60){(d)}
   \end{overpic}
\caption{Born cross sections of $e^{+}e^{-}\to
\pi^{+}\pi^{-}\psi(3770)$ (a), $e^{+}e^{-}\to
D_{1}(2420)^{0}\bar{D}^{0}+c.c.\to \pp\ddn$ (b), and
$e^{+}e^{-}\to D_{1}(2420)^{+}D^{-}+c.c.\to \pp\ddc$ (c). The
inner error bars are statistical errors, and the outer error bars
are the combined statistical and systematic errors.}
\label{Cpipipsi3770}
\end{figure}

For $e^{+}e^{-}\to \rho^{0}X(4013)$, we measure the upper limit of
the product of its Born cross section and the branching fraction
to $\ddb$, assuming $\mathcal{B}_{X(4013)\to D^{0}\bar{D}^{0}} =
\mathcal{B}_{X(4013)\to D^{+}D^{-}} = 0.5\mathcal{B}_{X(4013)\to
D\bar{D}}$. One may use Eq.~(\ref{xspspp}) with the efficiency
replaced by the simulation of the $X(4013)$ intermediate state,
and the number of signal events by the upper limit of the number
of signal events. The results are listed in Table~\ref{RCFX4013}.

For the process $e^{+}e^{-}\to \bar{D}D_{1}$, $D_{1}(2420)\to
X(D\pi^{+}\pi^{-} or D^{*}\pi)+c.c.$, the product Born cross section times
branching fraction of $D_{1}(2420)\to X$ can be
calculated using
\begin{equation}
  \sigma^{B}\times \mathcal{B}_{D_{1}(2420)\to X} =
  \frac{N^{\rm obs}}{\mathcal{L}_{\rm int} (1+\delta^{r})
  (1+\delta^{v}) \sum_{i,j}\epsilon_{i,j} \mathcal{B}_{i} \mathcal{B}_{j}},
\end{equation}
where $\epsilon_{i,j}$ is selection efficiency for $e^{+}e^{-} \to
D_{1}(2420)\bar{D}$ ($D_{1}(2420)\to D\pi^{+}\pi^{-}$, $D\to i$,
$\bar{D}\to j$) and the other variables as defined in
Eq.~(\ref{xspspp}). The numbers used and the cross section results
are shown in Tables~\ref{RCFD0D1},  ~\ref{RCFD0starD1}and~\ref{RCFDpD1} for neutral
and charged $D_1(2420)$, respectively. The cross sections as a
function of c.m. energy are also shown in Fig.~\ref{Cpipipsi3770}.

\section{Systematic uncertainty estimation}

The systematic uncertainties in the cross section measurements
mainly stem from the integrated luminosity, signal shape,
background shape, fitting range, tracking efficiency, photon
detection, kinematic fit, intermediate state decay branching
fractions, and ISR correction factor. The following items describe
how the uncertainties are estimated and the results are listed in
Tables~\ref{sys_err_psipp}, \ref{sys_err_X4013}, \ref{sys_err_D0}, \ref{sys_err_D0star},
and~\ref{sys_err_Dp}.

\begin{table}[htbp]
\scriptsize \caption{Summary of systematic uncertainties (in \%)
in $\sigma(e^{+}e^{-}\to \pi^{+}\pi^{-}\psi(3770))$ measurement.}
\label{sys_err_psipp}
\begin{tabular}{c c c c c c c c c c c c c c c c}
\hline \hline
     $\sqrt{s}$ (MeV) &4085.4 &4188.6 &4207.7  &4217.1  &4226.3  &4241.7 &4258.0  &4307.9   &4358.3  &4387.4   &4415.6  &4467.1   &4527.1  &4574.5   &4599.5\\
    \hline
    Luminosity                  & 1.0 & 1.0  & 1.0  & 1.0  & 1.0  & 1.0  &1.0   &1.0    & 1.0  & 1.0   & 1.0  & 1.0   & 1.0   & 1.0   & 1.0  \\
    Efficiency related          &4.4  &4.6   &4.3   &4.3   &3.9   &4.0   &4.3   &3.9    &3.8   &3.6    &3.4   &3.3    &3.2    &3.2    &3.0   \\
    Radiative correction        &1.34 &1.54  &2.13  &1.54  &0.56  &4.37  &1.11  &6.34   &0.60  &1.81   &2.11  &1.75   &0.66   &2.45   &0.18  \\
    Sinal shape                 &3.4  &3.4   &3.4   &3.4   &3.4   &3.4   &3.4   &3.4    &3.4   &3.4    &3.8   &3.8    &3.8    &3.8    &3.8   \\
    Background shape            &9.9  &9.9   &9.9   &9.9   &9.9   &9.9   &9.9   &9.9    &9.9   &9.9    &7.6   &7.6    &7.6    &7.6    &7.6   \\
    Fit Range                   &2.0  &2.0   &2.0   &2.0   &2.0   &2.0   &2.0   &2.0    &2.0   &2.0    &2.4   &2.4    &2.4    &2.4    &2.4   \\
    Signal region of double tag &5.3  &5.3   &5.3   &5.3   &5.3   &5.3   &5.3   &5.3    &5.3   &5.3    &2.0   &2.0    &2.0    &2.0    &2.0   \\
    \hline
    Total                       &11.7  &12.9  &12.9  &12.8  &12.6  &13.3  &12.7  &14.0  &12.5  &12.6  &9.9  &9.8  &9.7  &10.0  &9.6      \\
    \hline
    \hline
\end{tabular}
\end{table}

\begin{table}[htbp]
\caption{Summary of systematic uncertainties (in \%) in
$e^{+}e^{-}\to \rho^{0}X(4013)\to \pp\ddb$ measurement.}
\label{sys_err_X4013}
\begin{tabular}{c c c c}
\hline \hline
     $\sqrt{s}$ (MeV)      &4358.3   &4415.6   &4599.5\\
     \hline
    Luminosity                       & 1.0   & 1.0    & 1.0 \\
    Efficiency related               &2.6	 &2.7	  &2.7	\\
    Radiative correction             &5.6    &14.5    &5.9  \\
    Sinal shape                      &9.9    &0       &11.5 \\
    Background shape                 &3.2    &0       &4.6  \\
    Fit Range                        &4.5    &0       &4.4  \\
    Signal region of double tag      &3.3    &0       &12.1  \\
    \hline
    Total                            &13.4  &14.7  &19.0  \\
    \hline
    \hline
\end{tabular}
\end{table}

\begin{table}[htbp]
\caption{Summary of systematic uncertainties (in \%) for
$e^{+}e^{-}\to D_{1}(2420)^{0}\bar{D}^{0}, D_{1}(2420)^{0}\to D^{0}\pi^{+}\pi^{-} +c.c.$.}
\label{sys_err_D0}
\begin{tabular}{c  c  c  c c c c c c}
\hline \hline
   $\sqrt{s}$ (MeV)    &4307.9   &4358.3  &4387.4   &4415.6  &4467.1   &4527.1  &4574.5   &4599.5\\
    \hline
    Luminosity                  & 1.0   & 1.0  & 1.0   & 1.0  & 1.0   & 1.0   & 1.0   & 1.0 \\
    Efficiency related          &4.9	&4.9   &5.0	   &5.0	  &5.0    &4.9	  &5.0	  &4.9 \\
    Radiative correction        &6.34   &0.60  &1.81   &2.11  &1.75   &0.66   &2.45   &0.18   \\
    Sinal shape                 &7.21   &7.21  &7.21   &5.52  &5.52   &8.16   &8.16   &8.16  \\
    Background shape            &3.21   &3.21  &3.21   &1.23  &1.23   &7.35   &7.35   &7.35  \\
    Fit Range                   &0.94   &0.94  &0.94   &1.03  &1.03   &0.92   &0.92   &0.92   \\
    Signal region of double tag &2.64   &2.64  &2.64   &4.20  &4.20   &4.76   &4.76   &4.76  \\
%    Others                      & - & - & - & - & - & - & - & - \\
    \hline
    Total                       &11.6  &9.8  &9.9  &9.0  &8.9  &13.0  &13.3  &13.0     \\
    \hline
    \hline
\end{tabular}
\end{table}

\begin{table}[htbp]
\caption{Summary of systematic uncertainties (in \%) for
$e^{+}e^{-}\to D_{1}(2420)^{0}\bar{D}^{0}, D_{1}(2420)^{0}\to D^{*+}\pi^{-} +c.c.$.}
\label{sys_err_D0star}
\begin{tabular}{c  c  c  c c c c c c}
\hline \hline
   $\sqrt{s}$ (MeV)    &4307.9   &4358.3  &4387.4   &4415.6  &4467.1   &4527.1  &4574.5   &4599.5\\
    \hline
   Luminosity                  & 1.0   & 1.0  & 1.0   & 1.0  & 1.0   & 1.0   & 1.0   & 1.0 \\
    Efficiency related          &4.9	&4.9   &5.0	   &5.0	  &5.0    &4.9	  &5.0	  &4.9 \\
    Radiative correction        &6.34   &0.60  &1.81   &2.11  &1.75   &0.66   &2.45   &0.18   \\
    Sinal shape                 &5.34   &5.34  &5.34   &2.77  &2.77   &4.85   &4.85   &4.85  \\
    Background shape            &0.21   &0.21  &0.21   &1.79  &1.79   &7.35   &0.12   &0.12  \\
    Fit Range                   &2.70   &2.70  &2.70   &3.89  &3.89   &0.80   &0.80   &0.80   \\
    Signal region of double tag &4.93   &4.93  &4.93   &2.64  &2.64   &9.78   &9.78   &9.78  \\
%    Others                      & - & - & - & - & - & - & - & - \\
    \hline
    Total                       &11.2  &9.2  &9.5  &7.9  &7.9  &14.1  &12.3  &12.0    \\
    \hline
    \hline
\end{tabular}
\end{table}

\begin{table}[htbp]
\caption{Summary of systematic uncertainties (in $\%$) for
$e^{+}e^{-}\to D_{1}(2420)^{+}D^{-}+c.c.\to \pp\ddc$.}
\label{sys_err_Dp}
\begin{tabular}{c  c c c c c c c c}
\hline \hline
    Sources / $\sqrt{s}$ (MeV)   &4307.9   &4358.3  &4387.4   &4415.6  &4467.1   &4527.1  &4574.5   &4599.5 \\
    \hline
    Luminosity                   & 1.0   & 1.0   & 1.0   & 1.0   & 1.0   & 1.0  & 1.0   & 1.0 \\
    Efficiency related           &3.3    &3.3    &3.3    &3.3    &3.3    &3.3   &3.3    &3.3\\
    Radiative correction         &2.41   &0.24   &1.01   &0.98   &0.59   &3.44  &1.41   &1.51\\
    Sinal shape                  &7.40   &7.40   &7.40   &4.17   &4.17   &4.13  &4.13   &4.13 \\
    Background shape             &1.61   &1.61   &1.61   &0.81   &0.81   &3.83  &3.83   &3.83  \\
    Fit Range                    &1.74   &1.74   &1.74   &0.98   &0.98   &1.73  &1.73   &1.73  \\
    Signal region of double tag  &1.10   &1.10   &1.10   &1.33   &1.33   &5.23  &5.23   &5.23\\
%    Others                       & - & - & - & - & - & - & - & -\\
    \hline
    Total                        &8.9   &8.6  &8.6  &5.8  &5.7  &9.3  &8.7  &8.7       \\
    \hline
    \hline
\end{tabular}
\end{table}

(a) The uncertainty from the integrated luminosity measurement
using Bhabha ($e^{+}e^{-}\to e^{+}e^{-}$) scattering events is
estimated to be 1.0\% in Ref.~\cite{luminosity}.

(b) The systematic uncertainty caused by the signal shape is
estimated by varying the width of the signal by one standard
deviation of its world average value~\cite{PDG} in generating
signal MC sample. The difference in the cross section compared
with the nominal value is taken as the systematic uncertainty of
signal shape.

(c) The systematic uncertainty caused by the background shape from
MC simulated $\pi\pi\psi(3770)$, $D_{1}(2420)\bar{D}$ and $D_{1}(2460)\bar{D}$ is estimated by
changing the widthes of the $\psi(3770)$, $D_{1}(2420)$ and $D_{1}(2460)$ by one
standard deviations of the world average values~\cite{PDG} in
generating MC samples, the difference from the nominal fit values
are taken as the systematic uncertainties. These two uncertainties
are large because $e^{+}e^{-}\to \pp\psi(3770)$ and $e^{+}e^{-}\to
D_{1}(2420)\bar{D}$ have overlap in $M(D\bar{D})$ and
$M(D\pi^{+}\pi^{-})$, and the widthes of $\psi(3770)$ and
$D_{1}(2420)$ are not small, so the separation of the two
components is not easy in fitting, or in other words, there is
strong anti-correlation between these two components.

The systematic uncertainty from sideband selection is estimated by
changing the sideband windows by 10~MeV/$c^2$, the difference from
using the nominal mass windows are taken as the systematic
uncertainty. For $e^{+}e^{-}\to \rho^{0}X(4013)$, we change the
background shape from an exponential to a polynomial, and take the
difference as the systematic uncertainty of background shape.

(d) The systematic uncertainty from the fitting range is obtained
by varying the limits of the fit range by 20~MeV/$c^{2}$, the
difference is taken as the systematic uncertainty.

(e) In order to estimate the systematic uncertainty due to the
selection of signal windows in double $D$-tag, we repeat the whole
analysis by changing the signal regions from
$|\Delta{M}|<35$~MeV/$c^2$, $-6<\Delta\hat{M}<10$~MeV$/c^{2}$ to
$|\Delta{M}|<39$~MeV$/c^2$, $-8<\Delta\hat{M}<12$~MeV/$c^{2}$ for
$D^{0}\bar{D}^{0}$ mode; and from $|\Delta{M}|<25$~MeV/$c^2$,
$-5<\Delta\hat{M}<10$~MeV/$c^{2}$ to $|\Delta{M}|<29$~MeV/$c^2$,
$-7<\Delta\hat{M}<12$~MeV$/c^{2}$ for $D^{+}D^{-}$ mode. The
difference in the cross sections is taken as the systematic
uncertainty.

(f) The efficiency related systematic uncertainty includes the
uncertainties from MC statistics, particle identification,
tracking, $\pi^{0}$ and $K_{S}^{0}$ reconstruction, kinematic fit,
as well as the branching fractions of $D$ decays. Tracking
uncertainty in each charged track is 1\%~\cite{Tracking};
uncertainty in particle identification efficiency is 1\% per
track~\cite{Tracking}; uncertainty in the photon reconstruction is
1\% per photon~\cite{Photon}, and uncertainty in the $\pi^{0}$
reconstruction is 3\% per $\pi^{0}$~\cite{Photon}.

Uncertainty in the kinematic fit is estimated using the
track-parameter-correction method~\cite{kinematic fit}. The
difference between MC efficiencies before and after correction is
taken as the systematic uncertainty for kinematic fit (the
efficiencies from the track-parameter-corrected MC samples are
taken as the nominal ones).

The systematic error of $\mathcal{B}(D^{*+}\to \pi^{+}D^{0})$ is
taken as 0.74\%, those of $\mathcal{B}(D^{0}\to K^{-}\pi^{+})$,
$\mathcal{B}(D^{0}\to K^{-}\pi^{+}\pi^{0})$, $\mathcal{B}(D^{0}\to
K^{-}\pi^{+}\pi^{+}\pi^{-})$, and $\mathcal{B}(D^{0}\to
K^{-}\pi^{+}\pi^{+}\pi^{-}\pi^{0})$ are taken as 1.29\%, 3.60\%,
2.60\%, and 9.52\%, respectively, and those of
$\mathcal{B}(D^{+}\to K^{-}\pi^{+}\pi^{+})$, $\mathcal{B}(D^{+}\to
K^{-}\pi^{+}\pi^{+}\pi^{0})$, $\mathcal{B}(D^{+}\to
K^{0}_{S}\pi^{+})$, $\mathcal{B}(D^{+}\to
K^{0}_{S}\pi^{+}\pi^{0})$, $\mathcal{B}(D^{+}\to
K^{0}_{S}\pi^{+}\pi^{-}\pi^{+})$ are taken as 2.08\%, 3.01\%,
4.76\%, 3.86\%, and 3.53\%, respectively~\cite{PDG}.

The total efficiency related systematic uncertainty is the
quadratic sum of all the related individual uncertainties.

(g) ISR is simulated with {\sc kkmc}. The line shapes of the
processes understudy will affect the radiative correction factors
and the efficiencies. For $e^{+}e^{-}\to \pi^{+}\pi^{-}\psi(3770)$
and $e^{+}e^{-}\to \bar{D}D_{1}(2420)$, we first assume the observed
signal events originate from the $Y(4260)$ resonance to obtain the
efficiencies and ISR correction factors, then the measured line
shapes are used as input to calculate efficiencies and ISR
correction factors again. This procedure is repeated until the
difference between the two subsequent iterations is small enough
compared with the statistical uncertainties, and the difference is
taken as the systematic uncertainty.

Since we have no knowledge on the production of $e^{+}e^{-}\to
\rho^{0}X(4013)$, we assume $\rho^{0}X(4013)$ follows the
$Y(4260)$ or the $\psi(4415)$ line shape, the difference between
these two assumptions is taken as the systematic uncertainty.

Tables~\ref{sys_err_psipp}, \ref{sys_err_X4013}, \ref{sys_err_D0}, \ref{sys_err_D0star},
and~\ref{sys_err_Dp} summarize all the systematic uncertainties.
The overall systematic uncertainties are obtained by summing all
the sources of systematic uncertainties in quadrature by assuming
they are independent.

\section{Summary and discussions}

In this analysis, we analyzed $\EE\to \pp\ddb$ final states with
data samples collected at $\sqrt{s}$ = 4.09, 4.19, 4.21, 4.22,
4.23, 4.245, 4.26, 4.31, 4.36, 4.39, 4.42, 4,47, 4.53, 4.575, and
4.60~GeV.

We searched for $e^{+}e^{-}\to \pp\psi(3770)$ and the possible
$\pp\psi(1^{3}D_{3})$. We observed for the first time the process
$e^{+}e^{-}\to \pi^{+}\pi^{-}\psi(3770)$, but did not find
evidence for the $\psi(1^{3}D_{3})$. The Born cross sections of
$e^{+}e^{-}\to \pi^{+}\pi^{-}\psi(3770)$ are measured as shown in
Fig.~\ref{Cpipipsi3770}. These can be compared with the production
cross section of $\EE\to \pp\psi(1^3D_2)$~\cite{X3823}. If we take
 $\BR(\psi(1^3D_2)\to \gamma\chi_{c1})\approx
 \frac{250~{\rm keV}}{390~{\rm keV}}\approx 0.64$~\cite{BranchX3823},
the Born cross sections of $\EE\to \pp\psi(1^3D_1)$ is an order of
magnitude larger than those of $\EE\to \pp\psi(1^3D_2)$ at the
same c.m. energies~\cite{X3823}. The $e^{+}e^{-}\to \pp\psi(3770)$
line shape shows similar structure as in $\EE\to
\pp\psi(1^3D_2)$~\cite{X3823}, whether the events are from the
$Y(4360)$ or the $\psi(4415)$ or other resonances is still not
clear based on current statistics.

For the data points with enough statistics of $\EE\to \pp\pspp$,
we checked the Dalitz plot of the three-body system. We did not
observe significant structure in $\pi\pspp$ system, i.e., no the
signal of $Z_c$ is observed.

We also searched for the heavy-quark-spin-symmetry partner of the
$X(3872$), the $X(4013)$, via $e^{+}e^{-}\to
\pi^{+}\pi^{-}X(4013)$, with $X(4013)\to D\bar{D}$, no significant
signal of $X(4013)$ was observed in any data sample. The upper
limits of $\sigma(\EE\to \rho^{0}X(4013))\cdot
\mathcal{B}(X(4013)\to \ddb)$ are estimated as 5.1, 1.1, and
5.1~pb at $\sqrt{s}=4.36$, 4.42, and 4.6~GeV, respectively, at the
90\% C.L.

We also observed for the first time $e^{+}e^{-}\to
D_{1}(2420)^{0}\bar{D}^{0}+c.c.$ and $e^{+}e^{-}\to
D_{1}(2420)^{+}D^{-}+c.c.$ and measured the Born cross sections at
$\sqrt{s} = 4.31$, 4.36, 4.39, 4.42, 4,47, 4.53, 4.575, and
4.60~GeV, as shown in Fig.~\ref{Cpipipsi3770}. According to Zhao
{\it et al.}'s model to interpret the $Y(4260)$~\cite{Y4260asDD1},
a nontrivial structure will arise from the presence of the
$S$-wave $D_{1}(2420)\bar{D}$ threshold. In our results, there is no
fast rise of the cross sections above the $D_{1}(2420)\bar{D}$
threshold, instead, a broad structure at around 4.4~GeV is
observed.

\begin{thebibliography}{**}

\bibitem{BabarY4260} B. Aubert {\em et al.} (BABAR Collaboration),
Phys. Rev. Lett. {\bf 95}, 142001 (2005).

\bibitem{BabarY4360} B. Aubert {\em et al.} (BABAR Collaboration),

\bibitem{BELLEY4260} C. Z. Yuan {\em et al.} (Belle Collaboration),
Phys. Rev. Lett. {\bf 99}, 182004 (2007). Z.~Q.~Liu {\it et al.}
[Belle Collaboration], Phys.\ Rev.\ Lett.\  {\bf 110}, 252002
(2013).

\bibitem{BELLEY4360} X. L. Wang  {\em et al.} (Belle Collaboration),
Phys. Rev. Lett. {\bf 99}, 142002 (2007).  X.~L.~Wang {\it et al.}
[Belle Collaboration], Phys.\ Rev.\ D {\bf 91}, no. 11, 112007
(2015).

\bibitem{zc3900} M.Ablikim {\em et al.}, (BESIII Collaboration),
Phys. Rev. Lett. {\bf 110}, 252001 (2013).

\bibitem{X3823} M. Ablikim {\em et al.} (BESIII Collaboration),
Phys. Rev. Lett. {\bf 115}, 011803 (2015).

\bibitem{Highercharmonia} T. Barnes, S. Godfrey, and E. S. Swanson,
Phys. Rev. D {\bf 72}, 054026 (2005).

\bibitem{X3872} S. K. Choi {\em et al.} (Belle Collaboration),
Phys. Rev. Lett. {\bf 91}, 262001 (2003).

\bibitem{CDFX3872} D. Acosta {\em et al.} (CDF Collaboration),
Phys. Rev. Lett. {\bf 93}, 072001 (2004).

\bibitem{D0X3872} V. M. Abazov {\em et al.} (D0 Collaboration),
Phys. Rev. Lett. {\bf 93}, 162002 (2004).

\bibitem{BABARX3872} B. Aubert {\em et al.} (BABAR Collaboration),
Phys. Rev. D {\bf 71}, 071103 (2005).

\bibitem{PRD056004} J. Nieves and M. Pavon Valderrama,
Phys. Rev. D {\bf 86}, 056004 (2012).

\bibitem{PRD076006} C. Hidalgo-Duque, J. Nieves and
M. Pavon Valderrama, Phys. Rev. D {\bf 87}, 076006 (2013).

\bibitem{DecaywidthX4013} Feng-Kun Guo {\em et al.},
arXiv:1504.00861.

\bibitem{Y4260asDD1}Q. Wang, C. Hanhart and Q. Zhao,
Phys. Rev. Lett. {\bf 111}, 132003 (2013).

\bibitem{bepc2} M. Ablikim {\em et al.} [BESIII Collaboration],
Nucl.\ Instrum.\ Meth.\ A {\bf 614}, 345 (2010).

\bibitem{bes3yellow} D. M. Asner {\em et al.},
\Journal\IJMP{24}{499}{2009}.

\bibitem{boost} Z. Y. Deng {\em et al.},
HEP \& NP {\bf 30}, 371 (2006).

\bibitem{EvtGen} http://www.slac.stanford.edu/\~lange/EvtGen;
R. G. Ping {\em et al.}, Chinese Physics C {\bf 32}, 599 417 (2008).

\bibitem{PDG} K. A. Olive {\em et al.} (Particle Data Group),
Chin. Phys. C {\bf 38}, 090001 (2014).

\bibitem{Lundcharm} R. G. Ping {\em et al.},
Chin. Phys. C {\bf 32}, 599 (2008).

\bibitem{PYTHIA} http://home.thep.lu.se/torbjorn/Pythia.html

\bibitem{KKMC} S. Jadach, B. F. L. Ward, and Z. Was,
Comput. Phys. Commun. 130, 260 (2000);
Phys. Rev. D {\bf 63}, 113009 (2001).

\bibitem{upperlimit} J. Conrad {\em et al.},
Phys. Rev. D {\bf 67}, 012002 (2003).

\bibitem{vpcalculation} http://www-com.physik.hu-berlin.de/\~fjeger.

\bibitem{QEDcalculation} E. A. Kuraev and V. S. Fadin, Yad. Fiz. 41, 733-742 (1985).

\bibitem{luminosity} M. Ablikim al.(BESIII Collaboration),
Chin. Phys. C {\bf 39}, 093001 (2015).

\bibitem{Tracking}M. Ablikim {\em et al.} (BESIII Collaboration),
Phys. Rev. D {\bf 83}, 112005 (2011).

\bibitem{Photon}M. Ablikim {\em et al.} (BESIII Collaboration),
Phys. Rev. D {\bf 81}, 052005 (2010).

\bibitem{kinematic fit}M. Ablikim {\em et al.} (BESIII Collaboration),
Phys. Rev. D {\bf 87}, 012002 (2013).

\bibitem{BranchX3823} C. F. Qiao , F. Yuan, and K. T. Chao,
Phys. Rev. D {\bf 55}, 4001 (1997).

\end{thebibliography}
\end{document}
